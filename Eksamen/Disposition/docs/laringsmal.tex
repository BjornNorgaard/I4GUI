\section{Læringsmål}

\begin{itemize}
	\item [p.\pageref{sec:spm1}] Redegøre for principperne i .NET frameworket og dets overordnede arkitektur samt beskrive og vende programmeringssproget C\#.

	\item [p.\pageref{sec:spm2}] Designe og implementere programmer med en grafisk brugergrænseflade til Microsoft Windows platformen med brug af .NET frameworket og programmeringssprogene C\# og XAML.

	\item [p.\pageref{sec:spm3}] Anvende kontroller til opbygning af både modale og modeless dialoger, samt kunne anvende de forskellige layout panels.

	\item [p.\pageref{sec:spm4}] Anvende WPF's faciliteter til tegning af 2D grafik samt visning af billeder.
	
	\item [p.\pageref{sec:spm5}] Anvende styles og ressourcer.
	
	\item [p.\pageref{sec:spm6}] Anvende .NET frameworkets faciliteter til persistering af applikation- og brugerindstillinger samt til persistering af data i filer.
	
	\item [p.\pageref{sec:spm7}] Anvende data binding til at sammenknytte data i modellaget med deres præsentation i viewlaget.
	
	\item [p.\pageref{sec:spm8}] Redegøre for WPF's faciliteter til kommunikation mellem bagrundstråde og GUI-tråden i flertrådede programmer.
	
	\item [p.\pageref{sec:spm9}] Redegøre for arkitekturen for en Webapplikation.
	
	\item [p.\pageref{sec:spm10}] Designe og implementere Webapplikationer med en grafisk brugergrænseflade med brug af HTML5, CSS og javascript.
	
	\item [p.\pageref{sec:spm12}] Anvende et server side MVC framework til udvikling af Webapplikationer.
	
	\item [p.\pageref{sec:spm11}] Kunne anvende json-formatet i forbindelse med client-server kommunikation	
\end{itemize}